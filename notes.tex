\documentclass{article}
\usepackage[utf8]{inputenc}
\usepackage[table,xcdraw]{xcolor}
\usepackage[T1]{fontenc}
\usepackage{textcomp, gensymb}
\usepackage{amsmath}
\usepackage{physics}
\usepackage{ragged2e}
\usepackage{amssymb}
\usepackage{wasysym}
\usepackage{subcaption}
\usepackage[labelfont=bf]{caption}
\usepackage[margin=2cm]{geometry}
\usepackage[most]{tcolorbox}
\usepackage{adjustbox}
\usepackage{multirow}
\usepackage{marginnote}
\usepackage{float}
\usepackage{hyperref}
\hypersetup{
    colorlinks=true,
    linkcolor=darkgray,
    filecolor=darkgray,      
    urlcolor=darkgray,
    pdfpagemode=FullScreen,
    citecolor=darkgray,
    }
\urlstyle{same}
\usepackage{tablefootnote}
\usepackage{footnote}
\usepackage{graphicx}
\renewcommand{\familydefault}{\sfdefault}
\captionsetup{font={sf}}
\newcommand{\ee}{\mathrm{e}}
\newcommand{\ii}{\mathrm{i}}
\usepackage{svg}
\svgsetup{inkscapepath=svgsubdir}
\begin{document}
\title{Zagadnienie dynamiki wielomianów}
\author{Marta}
\date{\today}
\maketitle

\section{Wstęp}
These are my notes for the master's program.
Przyjmijmy, że w stanie stacjonarnym $t=0$ mamy do czynienia z układem opisanym przez wielomiany
\begin{equation}
    U(x) = \sum_{i=0}^2 a_i x^i
    =
    \begin{cases}
        U^{(\mathrm{C}, \; l)}(x), \quad x\in (-\infty; \; x_{\mathrm{C}}) \\
        U^{(\mathrm{B}, \; l)}(x), \quad x\in (x_{\mathrm{C}}; \; x_{\mathrm{T}})\\
        U^{(\mathrm{T}, \; l)}(x), \quad x\in (x_{\mathrm{T}}; \; \infty).
    \end{cases}
    \label{eq:stacjonarny-ogolna-postac}
\end{equation}
przy czym $a_i$ są stałymi, a $x_{\mathrm{C}}$ i $x_{\mathrm{T}}$ są punktami granicznymi.
Możemy też zapisać postać wielominaów w postaci kanonicznej przy następującymi zależnościach między postacią ogólną~\eqref{eq:stacjonarny-ogolna-postac}
\begin{equation}
    p \equiv \frac{-a_1}{2a_0},\\
    q \equiv \frac{a_1^2+ 4a_0 a_2}{4a_2}
\end{equation}
dla każdej wartości pary zasad $l=\mathrm{A-T, \; G-C}$ i form $\Tilde{F}$.
Możemy zatem zapisać wielomiany w postaci kanonicznej
\begin{align}
    U^{(\mathrm{C})}(x) = a_2^{(\mathrm{C})} (x - p^{\mathrm{(C)}})^2 + q^{(\mathrm{C})}, \quad x\in (-\infty; \; x_{\mathrm{C}}) \\
    U^{(\mathrm{B})}(x) = a_2^{(\mathrm{B})} (x - p^{(\mathrm{B})})^2 + q^{(\mathrm{B})}, \quad x\in (x_{\mathrm{C}}; \; x_{\mathrm{T}})\\
    U^{(\mathrm{T})}(x) = a_2^{(\mathrm{T})} (x - p^{(\mathrm{T})})^2 + q^{(\mathrm{T})}, \quad x\in (x_{\mathrm{T}}; \; \infty),
\end{align}
przy czym $U^{(\mathrm{\Tilde{F}})}(x)=U^{(\mathrm{\Tilde{F}}, \; l)}(x)$, $p^{(\Tilde{F})}=p^{(\Tilde{F}; \; l)}$, $q^{(\Tilde{F})}=q^{(\Tilde{F}; \; l)}$, $a_2^{(\Tilde{F})}=a_2^{(\Tilde{F}; \; l)}$ dla danego oscylatora $\Tilde{F}$ oraz pary zasad $l$.

\section{Wprowadzenie zależności od czasu}
Poprzez ciągłość będziemy mogli uzyskać pewne zależności między parametrami, które chcemy ustalić za zmienne niezależne i zależne.
\begin{enumerate}
\item Będziemy chcieli wprowadzić zależność czasową dla współczynników $a_2=a_2(t)$ dla każdej z form $F=\mathrm{C}, \; T$ i par zasad~$l$.
Tym samym, $a_2^{(\mathrm{B})}(t)$ będzie również zmienna w czasie, jednak będzie ona zależała od $a_2^{(\mathrm{C})}(t)$ i $a_2^{(\mathrm{T})}(t)$ poprzez warunki ciągłości w punktach granicznych $x_{\mathrm{C}}$ i $x_{\mathrm{T}}$ 
    \item Drugą możliwością będzie zapisanie punktów granicznych również jako zależne od czasu $x_{\mathrm{F}}=x_{\mathrm{F}}(t)$, jednak poprzez warunki ciągłości będą one zmiennymi zależnymi. 
    Pozostałe współczynniki $a^{\Tilde{F}}_2(t)$ będą zmiennymi niezależnymi.
\end{enumerate}

\subsection{Ciągłość}
Zapiszmy zatem warunki ciągłości 
\begin{align}
    a_2^{(\mathrm{C})}(t) (x_{\mathrm{(C)}} - p^{\mathrm{(C)}})^2 + q^{\mathrm{(C)}}
    =
    a_2^{(\mathrm{B})}(t) (x_{\mathrm{(C)}} - p^{\mathrm{(B)}})^2 + q^{\mathrm{(B)}},\\
    a_2^{(\mathrm{B})}(t) (x_{\mathrm{(T)}} - p^{\mathrm{(B)}})^2 + q^{\mathrm{(B)}}
    =
    a_2^{(\mathrm{T})}(t) (x_{\mathrm{(T)}} - p^{\mathrm{(T)}})^2 + q^{\mathrm{(T)}}.    
\end{align}


\end{document}