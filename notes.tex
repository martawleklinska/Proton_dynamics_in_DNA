\documentclass{article}
\usepackage[utf8]{inputenc}
\usepackage[table,xcdraw]{xcolor}
\usepackage[T1]{fontenc}
\usepackage{textcomp, gensymb}
\usepackage{amsmath}
\usepackage{physics}
\usepackage{ragged2e}
\usepackage{amssymb}
\usepackage{wasysym}
\usepackage{subcaption}
\usepackage[labelfont=bf]{caption}
\usepackage[margin=2cm]{geometry}
\usepackage[most]{tcolorbox}
\usepackage{adjustbox}
\usepackage{multirow}
\usepackage{marginnote}
\usepackage{float}
\usepackage{hyperref}
\hypersetup{
    colorlinks=true,
    linkcolor=darkgray,
    filecolor=darkgray,      
    urlcolor=darkgray,
    pdfpagemode=FullScreen,
    citecolor=darkgray,
    }
\urlstyle{same}
\usepackage{tablefootnote}
\usepackage{footnote}
\usepackage{graphicx}
\renewcommand{\familydefault}{\sfdefault}
\captionsetup{font={sf}}
\newcommand{\ee}{\mathrm{e}}
\newcommand{\ii}{\mathrm{i}}
\usepackage{svg}
\svgsetup{inkscapepath=svgsubdir}
\begin{document}
\title{Zagadnienie dynamiki wielomianów}
\author{Marta}
\date{\today}
\maketitle

\section{Wstęp}
Rozpatrywane są układy komplementarnie połączonych zasad azotowych wchodzących w skład DNA: adenina--tymina (A--T) oraz guanina--cytozyna.
Połączenia tych zasad bazuje na wiązaniach wodorowych, które polegają na oddziaływaniu atomu wodoru z atomami elektroujemnymi - w opisywanym przypadku: z atomami tlenu oraz azotu.
Przez geometrię układu dwóch zasad azotowych, dla atomu wodoru oddziałującego wodorowo z atomem elektroujemnym, ma on energetycznie możliwe położenia - związanego kowalencyjnie z jedną zasadą w stanie o niższej energii (położenie kanoniczne) oraz związanego kowalencyjnie z drugą zasadą (położenie tautomeryczne) w stanie o wyższej energii.
Między tymi dwoma stanami tworzy się bariera potencjału.
Slocombe et al.~\cite{Slocombe_CommPhys5y2022} oraz Godbeer et al.~\cite{Godbeer_PCCP17p13034y2015} wyekstrapolowali postać energii potencjalnej dla danego atomu wodoru (skrótowo nazywanego jako proton) dla kolejno par: G--C w postaci podwójnego potencjału Morse'a oraz A--T w postaci wielomianu czwartego stopnia.
Oba te profile potancjału można przybliżyć poprzez model dwóch oscylatorów harmonicznych oddzialonych paraboliczną barierą potencjału.
Na rysunku~\ref{fig:A-T-comparison-stationary} zostały ukazane profile potencjału dla profilu z pozycji~\cite{Godbeer_PCCP17p13034y2015} oraz przybliżoy model harmoniczny dla pary A--T.
\begin{figure}[htp!]
    \centering
    \begin{subfigure}{.495\textwidth}
        \centering
        \includegraphics[width = 1.0\textwidth]{graphics/true_sim/fourth_order_wave_funcs_AT.pdf}
        \caption{}
        \label{fig:A-T-potential-fourth-order}
    \end{subfigure}
    \begin{subfigure}{.495\textwidth}
        \centering
        \includegraphics[width = 1.0\textwidth]{graphics/model/model_AT.pdf}
        \caption{}
        \label{fig:A-T-potential-model}
    \end{subfigure}
    \caption{Wykresy profilu potencjału opisanego poprzez (a)~wielomian czwartego stopnia~\cite{Godbeer_PCCP17p13034y2015}; (b)~model harmoniczny; wraz z poziomami energetycznymi odpowiadających kolejnym stanom własnym}
    \label{fig:A-T-comparison-stationary}
\end{figure}
Na rysunku~\ref{fig:G-C-comparison-stationary} - porównanie potencjału danego od Slocombe et al.~\cite{Slocombe_CommPhys5y2022} oraz modelu dla G--C.
\begin{figure}[htp!]
    \centering
    \begin{subfigure}{.495\textwidth}
        \centering
        \includegraphics[width = 1.0\textwidth]{graphics/true_sim/morse_wave_funcs_GC.pdf}
        \caption{}
        \label{fig:G-C-potential-morse}
    \end{subfigure}
    \begin{subfigure}{.495\textwidth}
        \centering
        \includegraphics[width = 1.0\textwidth]{graphics/model/model_GC.pdf}
        \caption{}
        \label{fig:G-C-potential-model}
    \end{subfigure}
    \caption{Wykresy profilu potencjału opisanego poprzez (a)~wielomian czwartego stopnia~\cite{Godbeer_PCCP17p13034y2015}; (b)~model harmoniczny; wraz z poziomami energetycznymi odpowiadających kolejnym stanom własnym}
    \label{fig:G-C-comparison-stationary}
\end{figure}

\section{Zależność czasowa}
Przyjmijmy, że w stanie stacjonarnym $t=0$ mamy do czynienia z układem opisanym przez wielomiany
\begin{equation}
    U(x) = \sum_{i=0}^2 a_i x^i
    =
    \begin{cases}
        U^{(\mathrm{C}, \; l)}(x), \quad x\in (-\infty; \; x_{\mathrm{C}}) \\
        U^{(\mathrm{B}, \; l)}(x), \quad x\in (x_{\mathrm{C}}; \; x_{\mathrm{T}})\\
        U^{(\mathrm{T}, \; l)}(x), \quad x\in (x_{\mathrm{T}}; \; \infty).
    \end{cases}
    \label{eq:stacjonarny-ogolna-postac}
\end{equation}
przy czym $a_i$ są stałymi, a $x_{\mathrm{C}}$ i $x_{\mathrm{T}}$ są punktami granicznymi.
Możemy też zapisać postać wielominaów w postaci kanonicznej przy następującymi zależnościach między postacią ogólną~\eqref{eq:stacjonarny-ogolna-postac}
\begin{equation}
    p \equiv \frac{-a_1}{2a_0},\\
    q \equiv \frac{a_1^2+ 4a_0 a_2}{4a_2}
\end{equation}
dla każdej wartości pary zasad $l=\mathrm{A-T, \; G-C}$ i form $\Tilde{F}$.
Możemy zatem zapisać wielomiany w postaci kanonicznej
\begin{align}
    U^{(\mathrm{C})}(x) = a_2^{(\mathrm{C})} (x - p^{\mathrm{(C)}})^2 + q^{(\mathrm{C})}, \quad x\in (-\infty; \; x_{\mathrm{C}}) \\
    U^{(\mathrm{B})}(x) = a_2^{(\mathrm{B})} (x - p^{(\mathrm{B})})^2 + q^{(\mathrm{B})}, \quad x\in (x_{\mathrm{C}}; \; x_{\mathrm{T}})\\
    U^{(\mathrm{T})}(x) = a_2^{(\mathrm{T})} (x - p^{(\mathrm{T})})^2 + q^{(\mathrm{T})}, \quad x\in (x_{\mathrm{T}}; \; \infty),
\end{align}
przy czym $U^{(\mathrm{\Tilde{F}})}(x)=U^{(\mathrm{\Tilde{F}}, \; l)}(x)$, $p^{(\Tilde{F})}=p^{(\Tilde{F}; \; l)}$, $q^{(\Tilde{F})}=q^{(\Tilde{F}; \; l)}$, $a_2^{(\Tilde{F})}=a_2^{(\Tilde{F}; \; l)}$ dla danego oscylatora $\Tilde{F}$ oraz pary zasad $l$.

\subsection{Ciągłość}
Zapiszmy zatem warunki ciągłości 
\begin{align}
    a_2^{(\mathrm{C})}(t) (x_{\mathrm{(C)}} - p^{\mathrm{(C)}})^2 + q^{\mathrm{(C)}}
    =
    a_2^{(\mathrm{B})}(t) (x_{\mathrm{(C)}} - p^{\mathrm{(B)}})^2 + q^{\mathrm{(B)}},\\
    a_2^{(\mathrm{B})}(t) (x_{\mathrm{(T)}} - p^{\mathrm{(B)}})^2 + q^{\mathrm{(B)}}
    =
    a_2^{(\mathrm{T})}(t) (x_{\mathrm{(T)}} - p^{\mathrm{(T)}})^2 + q^{\mathrm{(T)}}.    
\end{align}
\begin{align}
    a^{(\mathrm{C})} (t) x_{(\mathrm{C})}^2(t) + b^{(\mathrm{C})} (t) x_{(\mathrm{C})} + c^{(\mathrm{C})}
    =
    a^{(\mathrm{B})} (t) x_{(\mathrm{C})}^2(t) + b^{(\mathrm{B})} (t) x_{(\mathrm{C})} + c^{(\mathrm{B})},\\
    a^{(\mathrm{B})} (t) x_{(\mathrm{T})}^2(t) + b^{(\mathrm{B})} (t) x_{(\mathrm{T})} + c^{(\mathrm{B})}
    =
    a^{(\mathrm{T})} (t) x_{(\mathrm{T})}^2(t) + b^{(\mathrm{T})} (t) x_{(\mathrm{T})} + c^{(\mathrm{T})}.
\end{align}
\section{Wprowadzenie zależności od czasu}
Poprzez ciągłość będziemy mogli uzyskać pewne zależności między parametrami, które chcemy ustalić za zmienne niezależne i zależne.
Będziemy mogli tego dokonać na kilka sposobów:
\begin{enumerate}
    \item Wprowadzenie zależności czasowej dla współczynników $a_2=a_^{(F)}2(t)$ dla każdej z form $F=\mathrm{C}, \; T$ i par zasad~$l$. 
            Układ równań dany warunkami ciągłości w punktach stałych w czasie $x_{\mathrm{C}}, \; x_{\mathrm{T}}$ dostarczy zależności współczynników $a_{2}^{(\mathrm{B})}(t), \; a_{2}^{(\mathrm{T})}(t)$ dla bariery oraz formy tautomerycznej bezpośrednio od współczynika $a_{2}^{(C)}(t)$.
            Znaczy to, że podczas kolejnych punktów czasowych, będzie się zmieniać niezależnie jedynie $a_{2}^{(\mathrm{C})}(t)$.
    \item Wprowadzenie zależności czasowej dla współczynników $a_2^{(\Tilde{F})}(t)$, $\Tilde{F}=\mathrm{C, \; B, \; T}$ oraz punktów $x_{\mathrm{C}}(t), \; x_{\mathrm{T}}(t)$.
            Układ równań dany warunkami ciągłości pozwoli wyznaczyć zależność zmieniających się punktów granicznych $x_{\mathrm{C}}(t), x_{\mathrm{T}}(t)$ od pozostałych współczynników~$a_{2}^{\Tilde{F}}(t)$.\\
            \\
            Jednak jako że równania te dają kwadratowe zależności, chcąc uzyskać jedynie wartości rzeczywiste, musimy stawić dodatkowy warunek aby wyróżnić kwadratowy danego równania był nieujemny.
\end{enumerate}

\subsection{}
Podczas każdego punktu czasowego zapisywane zostały zmieniające się współczynniki $a_{2}^{\Tilde{F}}$ oraz punkty graniczne.
Uzyskane zostały zatem ciągłe wielomiany (\textbf{gifs}).

\subsection{WDF}




\bibliographystyle{unsrt}
\bibliography{ref}

\end{document}